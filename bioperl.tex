\documentclass{article}

\usepackage {
 fullpage,fancyheadings,lscape,rotating,
 float,longtable,epsfig,psfig,layouts
}

\begin{document}

\begin{twocolumn}
\title{The Bioperl Project}
\author{The BioPerl Project Team, http://bio.perl.org}
\maketitle

\begin{abstract}
The Bioperl project is an open-source collaboration of biologists,
computation biologist, bioinformaticians, and computer programmers.
The mission of the project is to provide an accessible programming
toolkit for manipulating and managing life science information.
\end{abstract}

\section{Introduction and History}

In this paper we'll describe the history of the Bioperl project, the
initial goals and how we've met them, examples of Bioperl modules at
work in other projects, and future directions for the project.

In 1995 there were few programmimg toolkits for manipulating
biological data and results from sequence analysis and there were
certainly none in Perl.  The GCG package \cite{GCG} was used for
analyzing sequence data and determining phylogenetic trees.  Users
could automate analysis by writing scripts to pass files between GCG
programs.  However, a specific file format was required and
integration of results from different sequence analysis packages was
difficult to do computationally. Other tools such as the NCBI toolkit,
were being developed to provide applications for running sequence
comparision and database searching on a biologist's local machine.
The difficulty with all these packages is they required and intimate
knowledge of the command line arguments and which parameters to change
for desired.  Additionally the output from all the analysis packages
represented similar information but in different formats.  A user
attempting to compare results across environments found it difficult
to produce a simple summary table because of the sophisticated parsing
script he might have to write. \\

The Bioperl project grew out of a need to manipulate biological data
in many different file formats and sources.  Perl was an ideal
language because it can be used to write quick throw away scripts as
well as reusable code libraries.  While Perl does have limitations, it
provides an easy platform for writing simple bioinformatics solutions
for neophyte computer programming biologists and advanced
computational biologists alike.

\section{Project Goals} 

The goals of the Bioperl project are to provide simple interfaces for
manipulating life science data.  The project attempts to provide these
interfaces as compact perl library modules which are  

\subsection{Early History}

The origin of the bioperl project can be traced back to the Fall of
1995 in which a small group of people primarily made up of Steven
Brenner, Georg Fullen, Richard Resnick, and XXX began a mailing list
\cite{oldbioperlmaillist} to discuss building biological modules 
In the Fall of 1995, Steve Brenner along with other graduates (Georg
Fullen) of the University of Bielefeld course in bioinformatics
decided to build a set of perl modules that would help them unify a
representation of a biological sequence in all their programs.  They
wrote simple reusable operations for this object such as reverse
complement or protein translation of an exon sequence.  Additional
modules were begun to handle parsing different sequence file formats
to read sequences in from different sources and additionally write
them out again in various sequence file formats.  Other modules began
to accumulate after Steve and Georg contributing their own modules and
implementations of solutions to common bioinformatic problems.

\subsection{Modern History}
% Chris D: need Help here  -jason
In the XX of 199X money was obtained from ?? to buy an alpha linux
server for bioperl source code repository and mail/ftp/web server space.
Through the hard work of Chris Dagdigian - running out of his home for
X months before the Genetics Institute donated server room space and
bandwith to the project.  Since then we have grown to 5 projects
including the BioJava and BioPython 'language' projects and BioXML and
BioCORBA standards projects.  These will be discussed later in this
paper. 
\\ % Is this even relevant. -jason
The project was first coordinated in a 'benevolent dictator' style and
such great leaders as Steve Chervitz and Ewan Birney.  Later the
leadership was expanded to a core group of 4 developers and a sysadmin
(Ewan with Chris Dadigian, Hilmar Lapp, Heikki Lehvaslaiho and Jason
Stajich)


\section{Design}

\section{Development Process}

\section{Use Cases}

\section{Related projects}

\section{Contributors}

The primary contributors of the Bioperl project include the following
people along with others listed at http://bio.perl.org.
% I don't know where the contributor list should go or who should be
% on it, here is the start -jason

% Ewan Birney, David Block, Steve Brenner, Kris Boulez, Brad Chapman,
% Steve Chervitz, Chris Dagdigian, Georg Fullen, James Gilbert, Joseph
% Insana, Arek K., Hilmar Lapp, Heikki Lehvaslaiho, Aaron Mackey, 
% Matthew Pocock, Richard Resnick, Peter Schattner, 
% Jason Stajich, Lincoln Stein, Elia Stupka, Peter van Heuston, Mark Wilkinson

% other original contributors
% Lorenz? 
% (new) Arne?

\end{twocolumn}

\end{document}



