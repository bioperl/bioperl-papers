\documentclass{article}

% conventions -
%
% Interfaces are in italics
% Module names in Bold
% method names in bold italics
%
% when describing the bioperl objects, list all
% the applicable interfaces and objects at the beginning
% of the subsection like
% Objects: Object1 Object2
% Interfaces: Interface1 Interface2

\usepackage {
 fullpage,fancyheadings,lscape,rotating,
 float,longtable,epsfig,psfig,layouts
}

\begin{document}

\begin{twocolumn}
\title{The Bioperl Toolkit}
% Not sure how to do this - some people should be in the title?
\author{The BioPerl Core Team, http://www.bioperl.org}
\maketitle

\begin{abstract}

\bold{Motivation:} Perl is by far the most widely used programming
language in various domains of bioinformatics, which is commonly
attributed to the language being grasped quickly even by newcomers to
the field, and to Perl's many text-processing features. Programming in
Perl has been extremely successful for glueing together programs into
pipelines, converting between different formats, and extracting the
desired information from the output of analysis programs. However, a
significant part of the code of Perl Bioinformatics applications, and
consequently of the development time spent, consists of re-occurring
tasks, leading to a high extent of code duplication in the
Bioinformatics community. Perl programmers in the field are often
unfamiliar with software engineering concepts directed at alleviating
the problem, like generic modules re-used by specific applications, or
the overhead incurred by generalizing the program flow into re-usable
components is considered prohibitive.

\bold{Results:} Here we report about the Bioperl project, which is an
international open-source collaboration of biologists,
bioinformaticists, and computer scientists that has evolved over the
last 5 years into the most comprehensive library of
object-oriented Perl modules for manipulating and managing life
science information. BioPerl provides an easy-to-use, mature, stable,
and consistent programming interface to bioinformatics application
programmers. BioPerl has been successfully used to achieve
otherwise complex tasks with only few lines of code. We report about
the object and data model we conceived to represent various entities
one frequently has to deal with in the bioinformatics domain. The
BioPerl object model has been proven to be flexible enough to build
enterprise-level applications like ENSEMBL on top of it, while at the
same time keeping a steep learning curve even for inexperienced Perl
programmers. BioPerl is also capable of interoperating with other
programming languages like Python and Java through the evolving
BioCORBA bridge.

\bold{Availability:} BioPerl is available as open-source software free
of charge at http://www.bioperl.org/.

\end{abstract}

\section{Introduction}
At the time of inception of BioPerl in 1995 there were only few
programmimg toolkits for manipulating biological data and results from
sequence analysis programs, and there were in fact none available in
Perl, a scripting language that quickly gained widespread popularity
in the bioinformatics community for its efficient support of all sorts
of text processing tasks, and because neophyte computer programming
biologists without a background in computer science could easily grasp
the language.

While most compute-intensive analysis algorithms are implemented as
stand-alone programs in compiled languages like C for efficiency
reasons, putting together analysis pipelines for automating the tasks
of launching external programs, extracting results, and passing on
information to downstream programs requires mostly text-processing
facilities and convenient data structures.

Programming such tasks in Perl added powerful yet easy-to-learn
constructs for text-processing and information extraction as compared
to writing scripts that feed the output of one program to the next one
using packages understanding a common file format, e.g., the GCG
package \cite{GCG}. Such toolkits bear the difficulty that they
require an intimate knowledge of the programs' parameters, and how
they have to be changed to achieve the desired behaviour. Scripts were
inherently dependent on the particular program being run, not only on
the kind of algorithm. The output from different analysis programs for
the same purpose, like different alignment programs, represented
similar information but in different formats. Hence it was impossible
to create generic scripts treating external programs as pluggable
components.

The Bioperl project grew out of the following observations. First,
even though file formats of different analysis programs are different,
the represented information is the same. E.g., a pair-wise alignment
is always between two sequences and has common properties, like
length, score, fraction of identities, start and end on both
sequences, etc. 

Second, the number of data structures needed to represent the
information flow is limited, and common to most applications, like
sequences, annotation, features, and alignments. The language Perl
itself has no native way of defining abstract data types, unlike
e.g. C.

Third, a large number of operations performed are common between
applications, like parsing a data entity out of an input stream,
writing manipulated entities, filtering a stream of data entities to
screen for certain criteria, etc. 

This scenario naturally lends itself to the principle of object
oriented programming, which Perl emulates by modules. A module
corresponds to the notion of a `class' in object-oriented languages.
Hence, the mission of BioPerl is to provide an easy-to-use
object-oriented Perl library that encapsulates data structures and
operations commonly used in the life sciences as clean, generic, and
re-usable modules. The goal is that an application programmer using
BioPerl can focus on the problems specific to his or her application,
because BioPerl provides ready-to-use solutions for the those data
types and operations common to life science informatics. In addition,
we set out to design a data and object model that is easy to learn
even for Perl neophytes, and at the same time flexible enough to build
applications up to enterprise-level software.

Over time BioPerl has evolved into the most comprehensive toolkit for
bioinformatics in Perl. The programming interface is stable. Our
recent release of version 1.0 underlines its maturity.  In this paper
we report about BioPerl as of its production release 1.0.

The rest of the paper is organized as follows. First we describe the
data and object model we conceived, followed by an overview and
examples of the capabilities of the package. The third part of the
technical section elaborates on BioPerl's interoperability with
bioinformatics toolkits written in other programming languages. A
subsequent discussion ... We conclude with ...


\section{Systems and Methods}

\section{Implementation: Object and Data Model}

\section{Discussion}

\section{Conclusions}

\section{Acknowledgements}

\section{Old}

In this paper we'll describe the history of the Bioperl project, the
initial goals and how we've met them, examples of Bioperl modules at
work in other projects, and future directions for the project.

In 1995 there were few programmimg toolkits for manipulating
biological data and results from sequence analysis and there were
certainly none in Perl.  The GCG package \cite{GCG} was used for
analyzing sequence data and determining phylogenetic trees.  Users
could automate analysis by writing scripts to pass files between GCG
programs.  However, a specific file format was required and
integration of results from different sequence analysis packages was
difficult to do computationally. Other tools such as the NCBI toolkit,
were being developed to provide applications for running sequence
comparision and database searching on a biologist's local machine.
The difficulty with all these packages is they required and intimate
knowledge of the command line arguments and which parameters to change
for desired.  Additionally the output from all the analysis packages
represented similar information but in different formats.  A user
attempting to compare results across environments found it difficult
to produce a simple summary table because of the sophisticated parsing
script he might have to write. \\

The Bioperl project grew out of a need to manipulate biological data
in many different file formats and sources.  Perl was an ideal
language because it can be used to write quick throw away scripts as
well as reusable code libraries.  While Perl does have limitations, it
provides an easy platform for writing simple bioinformatics solutions
for neophyte computer programming biologists and advanced
computational biologists alike.

\section{Project Goals} 

The goals of the Bioperl project are to provide simple interfaces for
manipulating life science data.  The project attempts to provide these
interfaces as compact perl library modules which are  

\subsection{Early History}

The origin of the bioperl project can be traced back to the Fall of
1995 in which a small group of people primarily made up of Steven
Brenner, Georg Fullen, Richard Resnick, and XXX began a mailing list
\cite{oldbioperlmaillist} to discuss building biological modules 
In the Fall of 1995, Steve Brenner along with other graduates (Georg
Fullen) of the University of Bielefeld course in bioinformatics
decided to build a set of perl modules that would help them unify a
representation of a biological sequence in all their programs.  They
wrote simple reusable operations for this object such as reverse
complement or protein translation of an exon sequence.  Additional
modules were begun to handle parsing different sequence file formats
to read sequences in from different sources and additionally write
them out again in various sequence file formats.  Other modules began
to accumulate after Steve and Georg contributing their own modules and
implementations of solutions to common bioinformatic problems.

\subsection{Modern History}
% Chris D: need Help here  -jason
In the XX of 199X money was obtained from ?? to buy an alpha linux
server for bioperl source code repository and mail/ftp/web server space.
Through the hard work of Chris Dagdigian - running out of his home for
X months before the Genetics Institute donated server room space and
bandwith to the project.  Since then we have grown to 5 projects
including the BioJava and BioPython 'language' projects and BioXML and
BioCORBA standards projects.  These will be discussed later in this
paper.  
Formation of OpenBio...Hardware grant donation\\
\\ % Is this even relevant. -jason
The project was first coordinated in a 'benevolent dictator' style and
such great leaders as Steve Chervitz and Ewan Birney.  Later the
leadership was expanded to a core group of 4 developers and a sysadmin
(Ewan with Chris Dadigian, Hilmar Lapp, Heikki Lehvaslaiho and Jason
Stajich)


\section{Development Process}

The open-source model has lent itself to be extremely complementary to
the development process for a bioinformatics toolkit.  As
bioinformatics is very much a collaborative science, open source
development has encouraged a synergy of ideas.  With freely available
source, users of the toolkit can easily trace what modules do and how
to extend them for specialized cases.  Since anyone can contribute
their changes back to the toolkit other developers can take advantage
of their ideas and help maintain the code changes. 

\section{Design}

% Jason needs to rework!! --jason
Bioperl is an attempt to provide a fully Object-Oriented (OO) library of
perl modules for retrieving, storing, and manipulating life science
data especially in the sequence analysis corner of bioinformatics.
Attempts to make the toolkit fully OO have required the objects to be
defined as interfaces.  Since perl does not explictly have a
methodology for this we require all interface definitions in Bioperl
to be of the form ObjectI where 'I' indicates interface.  This can be
confusing when one tries to define a database interface without
indicating a subclass of the DBI module.  This document will italicize
all interface names and boldface all module names to help make clear
the distinction to the reader.  A preeceeding Bio:: indicating the
package namespace of all the Bioperl objects should be assumed for all
objects and interfaces discussed within this document.  
  
\par

We chose an OO design for the bioperl toolkit because it allows us to
properly assign tasks to objects and segregate functionality into
specialized objects rather than tackling all problems at once.
Utilizing interfaces one can separate the object's design from its
implementation.  This is particularly useful if one needs multiple
implementations of the same basic funcationality.  A good example of
this is the Sequence object in Bioperl. The \i{SeqI} interface
defines sequence objects which have associated sequence features.  It
is implemented by the \b{Seq} and \b{Seq::RichSeq} implementations
which serve as a simple and decorated implementation of the underlying
sequence.  Additionally the Ensembl project's VirtualContig object
implements the \i{SeqI} interface and represents a sequence
assemblies of a genome.  By implementing this in an Object-Oriented
bioperl methods can operate on the \b{Seq}, \b{Seq::RichSeq}, and
\b{EnsEMBL::Virtual::Contig} object in a uniform manner without
knowledge of which implementation is being processed.

\subsection{Sequence Objects}
Interfaces: \i{PrimarySeqI}, \i{SeqI}, \i{Seq::RichSeqI}
\\
Objects: \b{PrimarySeq}, \b{Seq}, \b{Seq::RichSeq}

Bioperl defines 2 types of Sequence objects, a \i{PrimarySeqI} and \i{SeqI}.
The \i{PrimarySeqI} (implemented by the class \b{PrimarySeq}) is a basic
sequence object containing a the DNA/RNA/Protein sequence as a perl
string.  The object also defines basic methods like 
\b{\i{length()}} - length of the sequence, 
\b{\i{moltype()}} - molecule type, 
\b{\i{revcom}} - reverse complement of a DNA dnastring, 
\b{\i{trunc()}} - a new PrimarySeq object truncated which is 
a specified subsequence of the parent sequence, 
\b{\i{substr()}} - similar to trunc() except it only returns the sequence string rather than an
entire PrimarySeq object.  
\par 
The SeqI inherits from PrimarySeqI and is implemented by the Seq
object.  The object defines methods for storing, retrieving, and
removing sequence features (SeqFeatureI) and annotations (Annotation).

\subsection{Sequence Features and Annotations}

Interfaces: SeqFeatureI, AnnotationI
Objects: SeqFeature::Generic, SeqFeature

Sequence Features are conceptualized annotations with locations on a
sequence.  Sequence feature can be attached to a sequence via the
add_SeqFeature method in Bio::SeqI 

\subsubsection{Sequence Feature Locations}
Interfaces: LocationI, Location::SplitLocationI,
Location::FuzzyLocationI, Location::CoordinatePolicyI
\\
Objects: Location::Simple, Location::Split, Location::Fuzzy,
Location::NarrowestCoordPolicy, Location::WidestCoordPolicy,
Location::AvWithinCoordPolicy
\\
Sequence feature locations are a difficult entity to model.  Start and
stop points of an annotation can be unknown, a feature can be located
as a collection of small pieces of non-contiguous locations, or a
feature could be referenced only relative to another feature's
location.  These are difficult concepts to model in software which is
oriented towards absolute values rather than fuzzy ranges.   The
Location packages attempt to localize this into a set of classes which
can handle simple (start/end/strand), fuzzy (unclear or relative
start/end points), and split( a collection of non-contiguous fuzzy or simple
locations).
  

\subsection{Sequence File Parsing}

The SeqIO package was written to provide flexible parsing of sequence
files.  Supported formats include Fasta, GenBank, EMBL, SwissProt,
GCC, AceDB, PIR, Raw sequence, SCF, and GAME XML. 
% nice to get PIR back in here when the module works again...

The objects are accessed by creating a new SeqIO object and iterating
through available sequences via the next\_seq() method.  A simple
demonstration of using the object to convert between sequence formats
can be found in Figure XX.



\section{Sequence Databases}

Interfaces: DB::SeqI, DB::RandomAccessI

\subsection{Local Sequence Databases}
Objects: Index::Abstract, Index::AbstractSeq, Index::Fasta,
Index::EMBL, Index::GenBank, Index::SwissPfam, Index::Swissprot

\subsection{Remote Sequence Databases}

Objects: DB::GenBank, DB::GenPept, DB::EMBL, DB::SwissProt

\subsection{Parsing Analysis Results}

\subsubsection{Sequence Database Searching and Report Parsing}

Objects: Tools::Blast, Tools::BPlite, Tools::Fasta,
Tools::HMMER
Interfaces: SeqAnalysisParserI, Factory::SeqAnalysisParserFactoryI

Parsing result files from Sequence database searches is a common need
in bioinformatics.  Typically BLAST and FASTA are used to generate
thousands of reports of which a researcher may only want to see a
small fraction.  Parsing these reports into queriable objects allows
programmers to write software to select only the reports that meet a
user's criteria.
\\
The Bio::SeqAnalysisParserI is a generic interface that defines one
method - \bold{\i{next\_feature}}.  This method returns a Bio::SeqFeatureI object
and allows a programmer to write software which iterates through an
analysis report which may be from Blast (NCBI or WU-Blast), Bl2seq,
PsiBlast, HMMER, or Fasta. 

\subsubsection{Multiple Sequence Alignment}

% Heikki to help write

\section{Sequence Variation Toolkit}

\subsection{Sequence Variation Toolkit}

\subsection{LiveSeq Toolkit} 
\section{Use Cases}

We believe the Bioperl toolkit is useful in a variety of settings,
both in small and large scale projects.  The following examples
show that Bioperl is powerful enough to be used in production
environments while suitably lightweight for use in individual research
labs.

\subsection{Ensembl}
Project Ensembl is a automated whole genome annotation system
coordinated as a joint project between the Eurpoean Bioinformatics
Institute (EBI) and the Sanger Centre in Hinxton, UK.    


\section{Related projects}

Since Bioperl's inception a number of different open-source toolkit
projects have been started.  

\subsection{BioPython}

\subsection{BioJava}

\subsection{BioCORBA}

\subsection{BioXML}

\section{Contributors}

The primary contributors of the Bioperl project include the following
people along with others listed at http://bio.perl.org.

%-- see the AUTHORS file --%

\end{twocolumn}

\end{document}
