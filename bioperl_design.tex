\documentclass{article}

\usepackage {
 url,graphicx,fancyheadings,fullpage,
 graphics
% float,layouts
}

\begin{document}
\pagenumbering{roman}
\begin{titlepage}

\title{Using Perl for Bioinformatics\\ Module 2: BioPerl}
\author{Jason Stajich, jason.stajich@duke.edu}

\end{titlepage}
\maketitle

\newpage

\tableofcontents
\newpage
\setcounter{page}{0}
\pagenumbering{arabic}

\section{This Document}
This document contains a description of the Bioperl toolkit from a
software design perspective.  Classes will be denoted like 
Interfaces are denoted \emph{Bio::Root::RootI} while implementations
are denoted \emph{\bf Bio::Seq}. Function names will be denoted {\bf \it
new()} while perl built-in functions denoted {\it split}.  

The objects and interfaces will be described in general
terms, for detailed lists of methods and their descriptions please see
the documentation embedded in the module files available through the
perldoc command or online at \url{http://bio.perl.org/Doc/Latest/Core/}.
\par
The bioperl toolkit is constantly evolving and this document describes a
snapshot of the current object model with disucssion of ideas in upcoming
releases.  The code released in the future will be a derivative of
the current code base but may not have the same limitations or
conventions as the 0.7 stable branch or 0.9 developer release
series.

\section{Bioperl Basics}

The Bioperl toolkit is one of the oldest open source bioinformatics
toolkit projects and the first Open Bioinformatics Foundation Bio\{*\}
project.  It was started in 1995 by a group of scientists tired of
rewriting BLAST and sequence parsers for various formats.  The
project has since grown to a 600+ subscriber email list, 10-15 main
developers coordinated by a core group of 4 developers, and deployment
in a variety of industry, academic, and govermental environments.  It
currently has modules focused on sequence format parsing, sequence
analysis report parsing, and annotation with an object model to
support most types of data formats in these realms.  It is an ever
growing and expanding project approaching a mature and stable 1.0
release by the end Q4 in 2001.

\par
Being an open-source project, anyone can jump in and make
contributions to improve the system.  Simply sign up on the mailing
list, introduce yourself and your idea of a new module or an
improvement to an existing one and we'll try and make suggestions to
help steer you in the direction to help sustain the consistency of the
toolkit and get your ideas implemented. 

\par

This course will dive into the basics of the Bioperl Toolkit, give you
experience using some of the modules, and help you approach some of
your own problems with a new set of tools at your disposal.  It is
assumed that you understand basic perl programing, know how to ``use''
modules in your code, and can take a stab at writing your own module.


\subsection{Bioperl Semantics}

Perl is not an inherently Object Oriented (OO) language.  In fact one has
to be fairly careful to make perl behave like an object oriented
system.  This requires a diligent programmer and some fairly
straightforward rules.  The Bioperl package takes advantages of the OO
constructs to create a consistent, well documented, and somewhat well
thought out object model for interacting with biological data in the
life sciences.  

\subsubsection{Bioperl Name space}
In perl namespaces are not enforced, but help the programmer and user
make sense of where things are installed.  The bioperl package
installs everything in the Bio:: namespace.  (In perl sub-name spaces
are the same as a directory structure and are separated by :: and are
equivalent to the way hierarchies are built in java with '.'.

\subsubsection{Interfaces}
For those familiar with software design, an Interface is a definition
for a set of methods that a class must implement.  Perl does not have
an explict system for forcing interfaces on to a class, but there are
some basic tricks one can play to enforce that a class implement an interface. 

\par
In Bioperl the \emph{Bio::Root::RootI} is the top-level Interface that all
Bioperl objects derive from.  It contains methods for debugging and
throwing errors ({\it \bf throw}, {\it \bf warn}), parsing named input
parameters ({\it \bf \_rearrange}), and the ability to build chained
constructor and destructor methods (for help with OO inheritance).

\par
Each class of data or function has its own interfaces defined.  For
example, Sequence data has the \emph{Bio::PrimarySeqI} interface while
annotated sequence data has the \emph{Bio::SeqI} interface which
is derived from the \emph{Bio::PrimarySeqI} interface.  Not all
classes in Bioperl have a properly defined inteface file as of the
0.7.x releases because sometime it takes some experience with the data
to derive a proper summary object to represent it. The 1.0 release
should have well defined interfaces for all ajor components of the toolkit. 

\section{Sequence Analysis}

Most of the work in Bioperl has been centered around supporting
sequence analysis.  Therefore a well defined set of objects has been
created to handle sequence, annotation, parsing and producing sequence
files, alignment, and analysis of sequence data.  

\subsection{Sequence Objects}

The \emph{Bio::PrimarySeqI} interface defines the basic methods for
interacting with sequence data.  Methods such as {\it \bf seq()} to
get and set the sequence string, {\it \bf trunc()} to return a truncated
version of the sequence, {\it \bf revcom} to get the reverse complement,
and {\it \bf translate} to get the protein for a RNA or DNA sequence.  The
interface is implemented by the \emph{\bf Bio::PrimarySeq} object
which represents sequences as in-memory strings.  The
\emph{\bf Bio::Seq::LargePrimarySeq} implements the
\emph{Bio::PrimarySeqI} interface as well but stores the sequence data
as a file since it is designed to handle sequences that are too large
to fit completely in memory.  Because of OO inheritance one can use
both of these implementations identically and not worry about how the
underlying methods are implemented.   

\par
The \emph{Bio::SeqI} interface inherits from \emph{Bio::PrimarySeqI}
and defines additional methods for storing and retrieving sequence
annotation in the form of sequence features (\emph{Bio::SeqFeatureI})
and annotation objects (\emph{\bf Bio::Annotation}).  The methods for
manipulating sequence features include {\it \bf top\_SeqFeatures()}, 
{\it\bf all\_SeqFeatures()}, {\it \bf feature\_count()}.  The {\it \bf
annotation()} provides access to get and set the associated Annotation
object.  The {\it \bf primaryseq()} hints at the delegation
pattern of the \emph{Bio::SeqI} objects which delegate all the sequence related
methods to an underlying \emph{Bio::PrimarySeqI} object.  The {\it \bf
species()} method allows one to associate a \emph{\bf Bio::Species}
object with a Sequence.  

\par The afforementioned \emph{\bf Bio::Seq::PrimarySeq} implements
the \emph{Bio::PrimarySeqI} interface and manages the large sequence
strings as files.  The \emph{\bf Bio::Seq::LargeSeq} is analgous to
the \emph{\bf Bio::Seq} object in that it manages the annotation but
delegates the sequence data specific methods to a \emph{\bf
Bio::Seq::LargePrimarySeq} object.  

\par
The \emph{Bio::Seq::RichSeqI} defines additional methods for
properly storing all the related sequence data in the GenBank/EMBL
data format.  It is implemented by the \emph{\bf Bio::Seq::RichSeq}
object.

\subsection{Sequence Annotation} 

The \emph{Bio::SeqFeatureI} interface represents the concept of an
annotated sequence feature which has a location on a sequence.  Most
sequence features are generic to be handled by \emph{\bf
Bio::SeqFeature::Generic}.  Essentially anything that can be stored in
the GFF format can be represented by this object.  However, more
complicated objects such as HSP's from BLAST or FASTA reports need to
contain information about the Query sequence location and the Subject
(Hit) location and so must be the composition of two SeqFeatures.  To
this end there are \emph{\bf Bio::SeqFeature::FeaturePair} objects for
paired Features and a subclass of these called \emph{\bf
Bio::SeqFeature::SimilarityPair} which can handle the notion that two
Features have similarity information about each other.  These object
will continue to evolve as the Bioperl project defines more general
objects to handle all types of sequence database searches.  

\par
The \emph{\bf Bio::Annotation} object handles the representation of
DBlinks, Literature References, and miscelaneous comments.
\emph{Annotations} are different than \emph{SeqFeatures} because
annotations do not have the concept of a location, they are a
description that apply to the entire sequence.  Annotations are also
more specialized to handle the concept of linking unique ids for a
sequence to databases.

\subsection{Sequence Input/Output}

Input and output or IO of sequence data can be the bane of biologists
trying to interact with bioinformatics tools.  The \emph{Bio::SeqIO}
system was designed to make getting and storing sequences to and from
the myriad of formats as easy as possible.  To created \emph{\bf
Bio::Seq} object from a sequence file in genbank, one can do this in
just a few lines of perl code using the Bioperl toolkit.
Figure~\ref{seqio} demonstrates this below.

\begin{verbatim}
use Bio::SeqIO;
my $seqio = new Bio::SeqIO('-format' => 'genbank', '-file' => 'filename.gb');
my $seqout = new Bio::SeqIO('-format' => 'embl', '-file' => '>filename.embl');
while( my $seq = $seqio->next_seq) { $seqout->write_seq($seq) }
\end{verbatim}
%$
\label{seqio}
\centerline{Figure~\ref{seqio}: A 3 line sequence converter}

\par The sequence parsing system will read rich sequence formats like
EMBL or GenBank and will create the associated features that are annotated
on a sequence.  Additionally the full set of features may be written
back out in the same or different format.  Obviously formats which do
not support the notion of features (FASTA, GCG, Raw) will not contain
as much information as an EMBL flatfile for the same sequence.
Table~\ref{formats} lists the available formats in the Bioperl 0.7
release series.

\begin{itemize}
\item Ace     - AceDB format
\item EMBL    - European Molecular Biology Laboratories sequence format
\item Fasta   - Bill Pearson's FASTA sequence database format
\item GAME    - Genomic Annotation Marker Environment an XML format
for BDGP
\item GenBank - NCBI sequnce format
\item GCG     - GCG sequence format
\item largefasta - for reading in FASTA files of very long sequences
(>100MB)
\item raw     - raw sequence
\item scf     - SCF chromatagram format
\item swiss   - SwissProt format
\end{itemize}
\label{formats}
\centerline{Table~\ref{formats}: Supported sequence formats by the \emph{Bio::SeqIO} system}

\section{Sequence Databases}

When processing many sequences, one often needs to store and retrieve
these sequences from databases rather than a collection of flatfiles.
The bioperl project provides interfaces for storing and and retrieving
from locally indexed databases (indexing speeds up the accessing of
the data but requires building a file larger than the data itsself)
and from remote sequence databases such as EMBL, GenBank, and SwissProt.

\subsection{Bioperl database interfaces}
The Bioperl DB interface is encapsulated in the \emph{Bio::DB::SeqI}
interface which describes mechanisms for retriving a sequence from a
database, either by accession number, id number, or as a
\emph{Bio::SeqIO} data stream.

\subsubsection{Local databases and Index files}
The \emph{Bio::Index::Abstract} object describes how one can build
local indexes of sequence data.  The indexing system uses the DBM
database system for building local indexes.  The Index objects defived
from the \emph{Bio::DB::SeqI} interface which allows the indexes to be
queries exactly like any other database.

\subsection{Remote databases}
One may also want to query a remote database such as NCBI's GenBank.
The module \emph{\bf Bio::DB::GenBank} allows queries for sequences
based on a single or multiple accession numbers or genbank ids.
Additionally there are similar modules written to query the NCBI's
GenPept database of proteins, EMBL's database of nucelotide and
protein sequence, and SwissProt's database of curated protein sequences.

\subsection{SQL databases}

The bioperl-db project (to be renamed biosql project) is a perl layer
on top of a relational database which allows access to a database of
sequence data.  The open-source database mysql was
chosen at the first implementation but a postgres, sybase, oracle, or
any other RBD could substitute.  The database is intended to be able
to store and fetch sequences and effectively represent annotation and
related information as rich as the GenBank/EMBL format.  Additional
work is being done .  More details available at the bioperl website
and the code is available under the bioperl-db cvs module name. 

\section{Sequence Analysis: Parsing and Running}

Parsing sequences and annotation is useful for a certain class of
analysis such as looking for simple patterns or small custom written
analyses, but typically one will want to rely on other software such
as BLAST or ClustalW to do sophisticated analysis.  Bioperl provides
mechanisms for both parsing the output from these programs and for
certain applications an interface for executing the programs within a
perl script.

\subsection{Parsing Analysis}

Parsing sequence analysis is critical for linking together results
from a collection of analyses and building suitable pipelines for
analysis.  Due to the nature of the bioperl development done by
different developers solving their own specific tasks, the interfaces
for analysis parsing are still evolving.  The current state of the
parsers are that each type is typically its own beast and it requires
the users to read the documentation before beginning to use a parser.
Hopefully this guide and the descriptions below will make this process
easier for new users wanting to parse their results.
  
\subsubsection{BLAST parsing}

BLAST is the staple of every bioinformatician's toolchest so a
suitable parser for the format is a necessity.  Currently two modules
attempt to address this need.  An older, complex module called
\emph{\bf Bio::Tools::Blast} is an extensible and robust object that
handles a wide variety of tasks including parsing and writing
HTML-ified Blast reports, and \emph{\bf Bio::Tools::BPlite} a simplier
less sophisticated parser that handles NCBI and WuBlast results.
Modules BPbl2seq and BPpsilite use objects from the BPlite system
handle bl2seq and psi-blast results respectively.   

\par The author recommends the BPlite module for those just starting
out with BLAST parsing.  The example in Figure~\ref{blastparsing}
should give the reader a good idea of how to get started.  

\begin{verbatim}
use Bio::Tools::BPlite;

my $parser = new Bio::Tools::BPlite('-file' => 'blast.report');
my $pvalue = 1e-3;
SBJCT: while( my $sbjct = $parser->nextSbjct ) {
    my  ($id) = split(/\s+/, $sbjct->name);
    while( my $hsp = $sbjct->nextHSP ) { 
	if( $hsp->P > $pvalue) {	
	    # skip Sbjcts that don't meet the minimum P value
	    print STDERR "skipping $id with a HSP with pvalue=", $hsp->P, "\n";
	    next SBJCT;    
	} 
	print "Sbjct length is ", $hsp->subject->length, "\n";	
    }

    # parse the ids if this a blast hit against an 
    # NCBI formatted library 
    my @ids = split(/\|/, $id);
    # in NCBI libs the last one is the accession number
    print "id is ", pop @ids, "\n";

}   
\end{verbatim}
%$
\label{blastparsing}
\centerline{Figure\ref{blastparsing}: Using \emph{\bf Bio::Tools::BPlite} for
blast parsing}

\subsection{Multiple Sequence Alignment Parsing}

Multiple sequence alignments are also a staple of bioinformatics
research.  Bioperl offers the \emph{Bio::AlignIO} system for reading
and writing MSA reports produced by a variety of sources include
ClustalW and GCG.  The system is analagous to the \emph{Bio::SeqIO}
system and provides a simple interface to parsing the data.  The {\it
\bf next\_aln()} method returns a \emph{\bf Bio::SimpleAlign} object
which encapsulates the MSA alignment.  One can ask for slices of the
alignment at certain columns, overal statistics, or for the references
to the individual sequences that make up the alignment.  Future work
in this area will establish an interface definition for this object
 and build a specialized version of the inteface for sequence
assemblies.  Figure~\ref{msaparsing} demonstrates the use of the
system to obtain alignments and query them.  Table~\ref{alignio}
describes the supported alignment formats the current
\emph{Bio::AlignIO} system.

\begin{itemize}
\item bl2seq
\item clustalw
\item fasta
\item mase
\item meme
\item msf
\item nexus
\item pfam
\item phylip
\item prodom
\item selex
\item stockholm
\end{itemize}
\label{alignio}
\centerline{Table~\ref{alignio}: \emph{Bio::AlignIO} supported formats}

\begin{verbatim}
use Bio::AlignIO;
my $alignio = new Bio::AlignIO(-format => 'clustalw',
			       -file => 'alignment1.aln');

while( my $aln = $alignio->next_aln ) {
 print "len=", $aln->length, 
 " # residures=", $aln->no_residues, " ",
 " percent id=", $aln->percentage_identity, "\n";
 print "seqs are :\n";
 foreach my $seq ($aln->each_seq) {
     print "\t '", $seq->display_id(), "'\n";
 }
 print "---\n";
}
\end{verbatim}
\label{msaparsing}
%$
\centerline{Figure~\ref{msaparsing}: The \emph{Bio::AlignIO} system in use}

\subsubsection{Gene prediction parsing}

Bioperl has also written parsers for a few standard gene prediction
programs: genscan and mzef.  These parsers can then produce \emph{\bf
Bio::SeqFeature::GeneStructure} objects which can represent all the
gene information returned by a prediction program.  Future work is
being done to write parsers for more formats and build gene objects
from annotations in sequence files.  For examples of using the parsers
the reader is directed towards the POD and the bioperl tutorial
available from the bioperl website.

\subsection{Generalize Parsing of Analysis}

The \emph{Bio::SeqAnalysisParserI} interface describes a simple way to
retrieve features from a parser for analyses that produce lists of
features.  It has a single iterator method {\it \bf next\_feature()}
which must return a \emph{Bio::SeqFeatureI} object.  This is a limited
interface which is still under developmemt as the bioperl project is
still trying to expand to handling a richer set of analysis types.
Some of the objects described do not comply to the
\emph{Bio::SeqAnalysisParserI} interface but may in the near future.
\par For those that do comply we have created a Factory object which
simplifies the code for a user.  The \emph{\bf
Bio::Factory::SeqAnalysisParserFactory} allows users to create an
analysis parser with just one line of code and handle the creation
of the specific parser object for you.  The code in
Figure~\ref{analysisfactory} should show a good example of how to use
it. 

\begin{verbatim}
use Bio::Factory::SeqAnalysisParserFactory;
# initialize an object implementing this interface, e.g.
$factory = Bio::Factory::SeqAnalysisParserFactory->new();
# find out the methods it knows about
print "registered methods: ",
    join(', ', keys(%{$factory->driver_table()})), "\n";
# obtain a parser object
my $inputobj = 'seq.genscan';
my @params = ();
my $method = 'genscan';
$parser = $factory->get_parser(-input=>$inputobj,
			       -params=>[@params],
			       -method => $method);
# $parser is an object implementing Bio::SeqAnalysisParserI
# annotate sequence with features produced by parser
while(my $feat = $parser->next_feature()) {
    print "feature is ", $feat->gff_string(), "\n";
    # if you had the sequence object
    #$seq->add_SeqFeature($feat);
    
}
\end{verbatim}
\label{analysisfactory}
\centerline{Figure~\ref{analysisfactory}:
\emph{\bf Bio::Factory::SeqAnalysisParserFactory} in action}

\subsection{Running Analyses}

\subsection{BLAST}
Support for running BLAST locally and remotely is built-in to bioperl
through the modules \emph{\bf Bio::Tools::Run::StandAloneBlast} and
\emph{\bf Bio::Tools::Run::RemoteBlast}.  These modules allow one to
either submit jobs to the NCBI blast queue or utilize a local blast
executable and databases.   Running BLAST locally requires setting
some environment variables which are documented in the POD included
with the module.

\subsection{ClustalW and TCoffee}
Multiple sequence alignment can be run locally on a set of sequences
using the \emph{\bf Bio::Tools::Run::Alignment::Clustal} and
\emph{\bf Bio::Tools::Run::Alignment::TCoffee} modules.  These modules
require the setting of an environment variable and local copies of the
executables which is documented in the POD.  

\section{Summary}

The bioperl toolkit is a collection of perl modules for bioinforatics
programming.  The dynamic nature of the toolkit's development may
cause some of the information in this document to be out of date.
Please join the mailing list if you find you want to learn more or
consider joining the development team.

\end{document}
