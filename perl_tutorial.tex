\documentclass{article}

\usepackage {
 url ,graphicx,fancyheadings,fullpage,
 graphics
% float,layouts
}

\begin{document}
\pagenumbering{roman}
\begin{titlepage}

\title{Using Perl for Bioinformatics\\ Module 1: Learning Perl}
\author{Jason Stajich, jason.stajich@duke.edu}

\end{titlepage}
\maketitle

\newpage

\tableofcontents
\newpage
\setcounter{page}{0}
\pagenumbering{arabic}
\section{Introduction and FAQ}
\subsection{Course Scope}

This one day course will attempt to teach you the basics of
programming in Perl including how to write simple scripts,
its applications to some common problems in life science research, and
pointers to sources of information to assist you in the future. 

\subsection{This document}
This document was prepared using the LaTeX typesetting system.  All
examples cited in this document are available in the directory where
the document source is found.
\par
This document is intended to be a short summary of the perl language,
but will not be as detailed as other books or the perl built-in
documentation about the perl language.

\subsection{Why use perl for bioinformatics ?}

Perl has seen widespread use in bioinformatics because of its ease of
use, extensiblity, and its ability to scale to complex tasks.  Being a
scripting language, new programmers do not have to wrestle with the
complexity of a compiled language and can quickly write scripts to
achieve the small - medium sized tasks that they need to accomplish
quickly.  Perl was built to provide easy string manipulation and has
simple I/O mechanisms making it ideal for formatting, searching, and
reorganizing data sets contained within files.  The {\bf P}ractical
{\bf E}xtraction and {\bf R}eporting {\bf L}anguage was built for
simplifying system administration tasks like {\it How many people
logged in over the last week on Machine X?}  The problem space of much
of bioinformatics is data manipulation, text processing, and simple
summary calculation which perl is perfectly suited to handle.

\subsection{Help I'm stuck and I can't debug}

When learning a new environment, be it programming a new language or
performing a new lab protocol, it is helpful to know where to search
for answers.  Perl has a built-in documentation system that can get
you started.  Additionally web resources and books have been published
which provide avenues of help to FAQ and difficult questions alike.
For questions of a biological nature, online forums such at newsgroups
bionet.software and bionet.biology.computational are useful as well as
the Bioperl project's mailing list.  Pointers to these web resources
are located in \emph{\it Resources} section.

\par
The built-in perl documentation system can be accessed by using the
command perldoc.  It is analagous to the man system in UNIX.  Typing
in perldoc and a Module name or -f Function Name will display the
manpage like documentation of the module or function name.  

\par
\begin{verbatim}
% perldoc perl
  show a listing of available documentation modules about perl.  
% perldoc perlsyn 
  describe the syntax of the perl language. 
% perldoc perldoc
  This will show the documentation for the perldoc system
% perldoc -f split
  This will show the documentation on the perl function 'split'
% perldoc File::Spec
  This will show documentation for the File::Spec module if installed
  on your system.
\end{verbatim}

\par
Spelling and case do matter.  If you aren't sure about the spelling of a
module's name one can use the -i flag to specify searching - for
example the following will display all the modules who start with File
\par
\begin{verbatim} 
% perldoc -i File
\end{verbatim}

Additionally, there is online help for all CPAN modules.  Go to
\url{http://www.cpan.org} and follow the directions for
searching for a particular module.

\section{Perl Basics}

Perl is an interpreted language and so all code that one writes runs
through the perl interpreter called 'perl' on most systems.  So one
writes perl code and runs it through the interepreter.  This tutorial
will familiarize you with the perl language and its syntax.
\par
 One of the first mantras you must come to understand about perl is
\emph{\it ``There's more than one way to do it'' (TMTOWTDI)}.  This has many
ramifications when trying to teach a perl class because often one
might only be taught how to do something one way, when in fact there
are a number of other ways to achieve the same outcome.
\par
 The first example of this has nothing to do with the perl language,
but is in the execution of scripts that are written in perl code.  One
can either specify the location of the perl interpreter in the format
{\tt \verb1#!1/usr/local/bin/perl}
OR one can omit this and run the script with the command
{\tt perl scriptname.pl}  
\par
On a typical UNIX system the perl interpreter is installed in the
directory {\tt ``/usr/local/bin/perl''}.  If we were in MS Windows system
the path to the perl interpreter might be ``\verb1C:\Perl\Perl.exe1''.  
\par
Additionally when specifying the name of the interpreter one can also
pass in command line arguments.  For example the \it -w \rm flag is
used to have perl warn the user about all warnings instead of
just suppressing non-fatal ones.     

\subsection{Hello World}

\begin{verbatim}
 #!/usr/local/bin/perl -w
 perl ``Hello World ''; 
\end{verbatim}
\label{hello}
\begin{center}{Figure~\ref{hello}: HelloWorld.pl}\end{center}


The code in Figure~\ref{hello} shows a a simple example of how to
print out the message ``Hello World''.  Line 1 is the specification of
where the perl interpreter is located.  Line 2 is the command to print
out some text, in this case the string ``Hello World'' followed by a
newline.

\subsection{Variables, Literals, Strings and Quotes}

Perl has support for 4 types of variables.
\begin{list}{$ \bullet $}{}
  \item Scalars - \$scalar\_variable
  \item Arrays  - @array\_variable
  \item Associate Arrays (Hashes) - \%hash\_variable
  \item GLOBS (Data streams) - \verb1*1STDOUT
\end{list}

Perl supports the literals: chars, integers, floating points,
references, and strings all within the scalar type.  Perl then casts a
scalar to an appropriate type based on context.  Example 2 in Figure~\ref{example2} shows how a variable containing a number can be treated
in either a numeric or character context.
  
\begin{verbatim}
 $a = 100;
 print ``Give me a $a dollar bill.\n''
 print ``Twenty times 10 = '', $a * 20, ``.\n'';  

 > perl example2.pl
 Give me a 100 dollar bill.
 Twenty times 10 = 2000.
\end{verbatim}
%$
\label{example2}
\begin{center}{Figure~\ref{example2}: example2.pl and output}\end{center}


Numbers can be declared as a string or a literal number.  The
following are all equivalent.
\begin{list}{$ \bullet $}{}
\item {\tt \$num = 100;}
\item {\tt \$num = ``100'';}
\item {\tt \$num = 1e2;}
\end{list}

Floating point numbers can be declared in the same way all other
numerics are declared.

\begin{verbatim}
$double = 1.0023;
$float  = '1.7890102';
$irrational = 1/3;
\end{verbatim}
%$ balancing for emacs

By putting the numeric in a string, it is stored as a string and
only casted to a particular numeric depending on the context it is
used.
\par
Characters are equivalent to strings of length 1 and so there is no
need to make a distinction between characters and strings as there is
in languages such as C or java.
\par 
In fact, the ease of using and manipulating strings in perl are one of
the strongest attraction to the language especially for creating quick
and dirty scripts to transform data.
\par
 Strings can be declared in a number of ways using different quoting
operators and functions.  Perl makes a distinction between the single
quote ' and the double quote ``.  The single quote is used to declare
a string as literally what is contained within the quotes without
outside references.  The double quote is used to declare a string but
process any references to other variables and use them to build the
string.  These are described in Example 3 found in Figure~\ref{example3}.
\par
Additionally strings can be built as composites of other strings in
much the same way the double quote operator does.  The '.' operator
takes 2 strings and returns a new string that is the concatenation of
the two.  This is equivalent to the '+' operator in java.  The sprintf
function may be familiar to C programmers as it allows one to produce
a formatted string using the universal C conventions.  More
information can be obtained from the perldoc system by issuing
perldoc~-f~sprintf.
\par
The backtick ` is a different beast, it is used to execute
programs in the local shell as in \linebreak {\tt \$datetime~=~`date`;}.
Executing external commands and programs will be covered in the
Advanced section.

\begin{verbatim}
# initialize the variable $a with the string literal 'yellow'
$a = 'yellow';

# initialize the $b to be a new string of $a concatenated with " submarine"
$b = "$a submarine";

# print out the value of $b and a newline (a return)

# print accepts a list of things to print separated by commas
# this is because print accepts a list to print - thus it is 
# accepting a list of items to print, and everything can be 
# converted to a string

print $b, "\n";

# help show the difference between single quote (') 
# and double quote (")
$c = 'green is the opposite of $a';

print $c, " is the single quoted message\n";

# concatenate 3 strings together 
print "Curious George and the man with the ". $a . " hat" . "\n";

print "---------\n";
print "make sure you use double quotes when printing \nspecial ",
'characters like \'\n\' and \'\t\' '.
'or else they show up literally', "\n\n";

print "if you want to print a double quote \neither escape it \"",
    ' or use single \' to print it - "', "\n";

\end{verbatim}
\label{example3}
\begin{center}{Figure~\ref{example3}: example3.pl}\end{center}

\subsection{Arrays and Lists}

Arrays and lists are ordered collections of items.  The different
between a list and an array is academic, an array is a variable that
contains a list while a list is simply an anonymous listing of items.
The first part of the example in Figure~\ref{example4} might make it
clearer.  Arrays and lists do not need to contain all the same type of
items, an array can contain numerics, scalars, or references.
However, arrays cannot contain other arrays directly - perl always tries to
flatten arrays of arrays into a single array.  If one is interested in
multi-dimensional arrays (arrays of arrays) it is necessary to use the
notion of an array reference.  This is declared with the [] operators
around the list one wants to declaring as an array reference.  Perl
also uses the special operator $\backslash$@array to indicate a
reference to an already existing array.  The operators [@array] and
$\backslash$@arrays are then equivalent for all intensive purposes.

Arrays can be indexed via the the [] operator.  To get the 1st item in
a list we use \$array[0] (arrays start at 0 like in C).  There are a
number of functions for use with arrays.  The pop and push methods
treat a list like a stack by removing and adding elements to the end
of the array.  shift and unshift are their cousins operating on the
front of the array instead.  The splice function can operate anywhere
in the array and can insert or remove elements anywhere in the array.

\begin{verbatim}
# a list - it is not stored in a variable so it is anonymous
(1, 200, '90.5', 'english');
# an array
@array = ('a', 'b', 'c');
print "1st array element is '", $array[0], "'\n"; 

# can mix elements in an array
@array = ('a', 1, '89', 2.34);

# some array functions

# build an array by taking things that are 
# separated by whitespace
@array = qw(1 dog barked);

# elements of the array can be joined together 
# with the 'join' command which will join 2 
# elements together
print "elements are -- ", join(' ', @array), "\n";

# arrays are flattened in perl
@array = (100, 200, (220, 230, 280), 300);

print "element list is ", join(',', @array), "\n";

# array references allow us to build arrays of arrays
@matrix = ( [ qw( 67 78 98) ],
	    [ qw( 99 95 82) ], );

print join("\t", qw(Test1 Test2 Test3)), "\n";

# will discuss loops and conditionals later on, but 
# here is how to see each 'row' in the matrix and
# print the average score\n";
foreach my $row ( @matrix ) {    
    # use @$row to dereference an array reference
    $total = 0;
    foreach my $col ( @$row ) {
	$total += $col;
    }
    $average = $total / scalar @$row;
    print join("\t", @$row), "\t--> ",  $average, "\n";
}

# pop and push work on the end
push @array, 1;
my $one = pop @array;
die unless ($one == 1);

# shift and unshift work on the front
unshift @array, 'bee';
my $two = shift @array;
die unless ($two eq 'bee'); 

# splice can operate anywhere on the array
# remove the elements at position 1 and 2
print "array is ", join(',', @array), "\n";
my (@elements) = splice(@array, 1, 2);
print "array is ", join(',', @array), "\n";
# get out the elements at position 1 and 2
# and replace them with new elements
my (@elements) = splice(@array, 1, 2, (109, 207));
print "array is ", join(',', @array), "\n";

\end{verbatim}
\label{example4}
\begin{center}{Figure~\ref{example4}: example4.pl}\end{center}
%$
\par

In perl everything can be treated like a list.  This derives in part
from the mantra \emph{\it TMTOWTDI}.  The fact that everything can be
treated as a list makes argument passing very simple - one only has to
expect a list of arguments and can process them based on what is
expected.  It also lets you be an extremely lazy programmer and since
perl does not enforce strong typing of variables one can get into
trouble.  There are mechanisms for supporting strong typing in certain
contexts, but it puts the burden of checking on the programmer.

\subsection{Associative Arrays or Hashes}

Associative arrays or hashes allow one to index a list of items on
arbitrary keys rather than numerical indexes. They are equivalent to
the concept of hashtables in Java or C++ and provide fast lookups.
The examples in Figure~\ref{example5} show how one can use an
associative array to index things in different ways.  To get or store
an item on a specific key one uses the \{ \} operators to do the lookup. 
If on wanted to store an array hashed to a specific key, we would
operate in the same way as with arrays - by storing an array reference
at the location.  

\begin{verbatim}
# associative arrays can be declared in different ways
# declared and initialized at the same time
my %hashmonths = ( 'Jan' => 1, 'Feb' => 2, 'Mar' => 3);

# declared, then initialized
my %hashdays;
$hashdays{'Sun'} = 1;
$hashdays{'Mon'} = 2;
$hashdays{'Tue'} = 3;
$hashdays{'Wed'} = 4;

# a reference to a hash
my $hashref = { 'Jul' => 7, 'Aug' => 8};

# get all the keys in a hash
my @keys = keys %hashmonths;
print "keys are ", join(',', @keys), "\n";
# get all the values in a hash
my @values = values %hashdays;
print "values are ", join(',', @values), "\n";
# notice that insert order is not necessarily preserved

# iterate through data
while( my ($day,$val) = each %hashdays ) {
    print "$day = $val\n";
}

use Tie::IxHash;
tie %orderedhash, "Tie::IxHash";

$orderedhash{'Guitar'} = 'B.B.King';
$orderedhash{'Trumpet'} = 'Dizzy Gilespie';
$orderedhash{'Accordian'} = 'Weird Al';

foreach my $key ( keys %orderedhash ) {
    print "Instrument '$key' played by $orderedhash{$key}\n";
}

# store an array reference in a hash entry
$orderedhash{'Guitar'} = [ 'BB King', 'Eric Clapton'];

# add some more people
push @{$orderedhash{'Guitar'}}, 'Jimmy Page';

# store a hash ref in a hash
my %hash3 = ( 'canine' => { 
		  'domestic'  => [ 'hounds', 'labs', 'retrievers'],
		  'wild'      => [ 'grey', 'siberian' ]},
	      'feline' => { 
		  'domestic' => [ 'calico', 'tabby', 'siamese', 'persian' ],
		  'wild'     => [ 'panther', 'lynx', 'bobcat', 'lion', 
				  'tiger' ]});

foreach my $animal ( keys %hash3 ) {
    print $animal, "\n";	
    while( my ($domain_name,$domain) = each %{$hash3{$animal} } ) {
       print "\t$domain_name\n";
       foreach my $type ( @$domain ) { print "\t\t$type\n"; } }
}
\end{verbatim}
\label{example5}
\begin{center}{Figure~\ref{example5}: example5.pl}\end{center}

\subsection{A Brief Discussion of Truth}

The examples in this document utilize the flexibility of
perl with regard to truth without informing the reader.  In perl,
everything can be tested for whether it is true or false.  The rule is
that non-zero, non-empty values are considered true, everything else
is false.
\par
In the case of a loop such as a foreach loop (will be discussed in the
next section) the loop will iterate through all the items in the list.
In the case of a while loop, the loop will iterate until the condition
is false - the list is empty.  

\begin{verbatim}
if( @a ) { # true of @a is non empty}
if( %a ) { # true if %a is non empty}

while( @a ) { # must remove items from @a to prevent infinite loop}
\end{verbatim}

\subsection{Conditionals and Loops}

\par
Perl has the standard constructs for conditionals.

\begin{list}{}{}

\item if..elsif..else
\item logically opposite unless..elsif..else

\end{list}

Loops are supported in perl with the following constructs.

\begin{list}{}{}

\item for(initialize; conditional; iterate) \{ \}
\item while( conditional ) {}
\item do \{\} while(conditional) - 
\item foreach \$itemptr ( @items ) \{ \}

\end{list}

\subsection{Built-in perl functions}
\subsubsection{String functions}
Since a large amount of data manipulation in perl centers around
strings, there are quite a few built in functions for operating on
strings.  
\begin{list}{$ \bullet $}{}

\item {\it split} splits a string into an array based on a separator.  It
takes as arguments, the separator pattern, the string, and optionally
the number of items to split the string into and will return an array
of items.  If no number is entered for this number then {\it split}
will attempt to break up the string where all separators exist.
Otherwise it will only break up the string into the first N-1 items
and store the rest of the string in the Nth item.  The separator value
can be a regular expression (see Section on Regular Expressions) a
string.  So one can pass in \verb1/\s+/1 to match any whitespace or
':' to match semi-colons as separators in a list.

\item {\it join} joins a list of items into a single string separated
by a specified string.  It takes as arguments the separator string and
a list of items to join together and returns a single string.  It is a convient way of viewing and manipulating data when
one only wants to perform an operation between all consecutive
pairs in a list - the ``fencepost'' problem.   

\item {\it substr} returns a substring of a specified string.  It takes
as arguments an input string, the offset point in the string
(strings start at 0 like arrays), and optionally the length of the substring to
extract.  If no length is specified then the substring returns
everything from offset to the end of the string.  The offset can be
negative to indicate starting at the end of the string.  substr also
takes an optional 4th argument which specifies a replacement string to
replace the substring with in the query in an analagous way to the
splice function on arrays.

\item {\it sprintf} and {\it printf} can be used to create and print
formatted strings.  They both use the
standard C syntax of a formatting string and then a list of items to
use in the formatting.  {\it sprintf} returns a string while {\it
printf} takes an optional 1st argument which specifies the filehandle
to print the string to (STDOUT is implied when no filehandle is
specified). 

\item {\it print} is a simplified form of {\it printf}.  It
accepts an optional filehandle and a list of items to print to the
filehandle.  If no items are specfied {\it print} will print the
implicit \$\verb1_1 variable (see the Section on Special Variables) 

\item {\it chomp} and {\it chop} are used to remove trailing input
record (in most cases this is ``\verb1\n1'').  It is most useful when reading
in data from an input loop and the newline separator is not already
removed. {\it chop} is the unsafe version of chomp in that it will
delete the last character of a string regardless of whether or not it
is the record separator.  chop and chomp will also operate on a list
by processing each item in the list.

\item {\it lc} and {\it uc} respectively will return a lowercase and uppercase
version of an input string. {\it lcfirst} and {\it ucfirst} only lower
and uppercase the first letter in a string.

\item {\it index} and {\it rindex} will find the first occurance of a
substring within a specified string.  They take as arguments the query
string, the sub string, and optionally the offset to start from.  {\it
index} starts at the beginning of the string while {\it rindex} starts
searching from the end.

\item {\it reverse} will return a copy of the input string exactly reversed. 

\item {\it \verb1q{STRING STRING}1 } will return a string literally as
what is typed between the \{ \} (which can be subsituted for by any
other pairing of separator ( (), \verb1/ /1, $\vert $ $\vert $).

\item {\it chr} and {\it ord} return the character and ascii value
respectively for an input number or character.  {\it chr} will return
the character value for an ascii value and {\it ord} will return the
appropriate ascii value for a character. 

\end{list} 

\subsubsection{Array functions}

\begin{list}{$ \bullet $}{}

\item {\it scalar} returns the number of items in an array.  The
special operator \verb1$#1 returns the index of the last item in the
array, effectively scalar @array - 1;
%$
\item {\it map} evaluates an operation block and operates on each item
in a list.  It takes as arguments a BLOCK and list of items to operate
on and returns a list of values for each of the operations.  A good
example is @chars = map(chr, @nums) which will return a list of
characters for a given set of ascii values.
 
\item {\it grep} will evaluate a given block or regexp on all items in
list and return the items which evaluated to true.  It behaves much in
the same way UNIX grep does except that it is operating on a list of
items rather than a file.

\item {\it reverse} behaves exactly as reverse on a string works
returning an entire array reversed.

\item {\it sort} returns a list sorted based on the input sort
function and the array.  If no function is passed the default
alphanumeric least to greatest is applied.  

\item {\it splice}, {\it push}, {\it pop}, {\it shift},
{\it unshift} were described earlier and only operate on a real array
of data not an anonymous list.

\end{list}

\subsection{Input/Output}

Perl I/O is by default incredibly easy and can be tweaked in
sophisticated ways to handle non-default behavior.  The main operators
and functions involved in Perl I/O are the {\it open} function and $
<> $ operator.    

The {\it open} function is used to open filehandles.  The syntax is to
specify the name of the filehandle and the filename or system
operation on wishes to perform.  The examples in Example 6 in Figure~\ref{example6} show how this can be done. 

For more advanced users one can either open handle to a program which
will be producing output or accepting input.  To manipulate a program
which will be producing and receiving output one must use the module
IPC::Open3 which requires a fairly sophisticated understanding of how
perl IO works.
 
\begin{verbatim}
open(IN, "< example6.pl"); # read this script
# and print it to STDOUT...
while(<IN>) {
    print;
}
close(IN);
print "-----\n";
open(PERLDOC, "perldoc -f open |");
# get the perldoc page on the open function
while(<PERLDOC>) {
    print;
}
close PERLDOC;
print "-----\n";
open(WC, "| wc");
print WC "a simple mesage\nwith 2 lines 10 words 52 characters\n";
close WC;
\end{verbatim}
\label{example6}
\begin{center}{Figure~\ref{example6}: example6.pl}\end{center}

\subsection{Regular Expressions}

Regular expressions are the bread and butter of Perl.  The popularity
of the perl regexp syntax and ease of use has spawned the GNU project
to create a perl regexp library to allow easy porting of regexps to
different language environments.  We will walk through a few examples
in this text and the in-class tutorial but I commend the reader to
some of the numerous web resources and books that are available for
perl.

\par
Regular expressions are usually invoked within the perl pairing
\verb1/ /1.
There are a few special characters in regular expressions which have
special meaning.  
\begin{list}{$ \bullet $}{}
\item \verb1\s1 - match whitespace (any tab, space, newline)
\item \verb1\S1 - match anything EXCEPT whitespace
\item \verb1\d1 - match any digit (0-9)
\item \verb1\D1 - match anything EXCEPT a digit
\item \verb1\w1 - match a word (any alphanumeric) 
\item \verb1\W1 - match anything EXCEPT an alphanumeric

\end{list}

Additionally quantifier operators are defined
\begin{list}{$ \bullet $}{}
\item \verb1*1 - match expression 0 or more times
\item \verb1+1 - match expression 1 or more times
\item \verb1?1 - match expression 0 or 1 times
\item \verb1{n}1 - match exactly n times
\item \verb1{n,}1 - match at least n times
\item \verb1{n,m}1 - match at least n but not more than m times

\end{list}

Within the regular expression the [item] operators allow the programmer
to specify a list of items that are allowed to match, providing some
flexiblity for matching multiple patterns. 

Perl defines some assertion operators:
\begin{list}{$ \bullet $}{}
\item \verb1^1 - assert that match begins at start of string
\item \$ - assert that match ends at end of string

\end{list}

Pattern matching can be initiated with a few different operators.
Enclosed by the \verb1/ /1, a pattern can be done on the implict
variable \verb1$_1 or by using the \verb1=~1 (positive assertion) or
\verb1!~1 (negative assertion) operators.  Figure~\ref{onelinermatch}
demonstrates these.  

%$
When patterns are enclosed in parentheses in a regular expression the
regular expressions stores the match.  If it is the first set of
parentheses, the data is stored in the variable \$1, if it is the
second set then \$2, and so on.  One can also write a one-liner
(one-liners are a Perl staple) to match and store the values in
parens.  See Figure~\ref{onelinermatch} for an example.

\begin{verbatim}
if( /^(\d+)/ ) { # match leading numerics in $_
 print ``$1 is the match\n'';
}
# match trailing numerics w/ or w/o whitespace 
# in variable $line and store in variable $num
my ($num) = ( $line =~ /(\d+)\s*$/ ); 
                               
if( $str !~ /\S+/ ) { # test if str does not contain any non-whitespace 
}
\end{verbatim}
\label{onelinermatch}
\begin{center}{Figure~\ref{onelinermatch}: One-Liners for pattern matching}\end{center}

So let's write a regular expression which matches the zip code from a
list of addresses.  See Figure~\ref{example7} for the code.

\begin{verbatim}
my @data = ( '1 Infinite Loop, Cuperteno, CA, 95014-2084',
	     '201 Vassar St, W59-200, Cambridge, MA 02139');

foreach my $address ( @data ) {
    chomp;
    if( $address =~ /(\d{5}(-\d{4})?)$/ ) {
        print "zip is $1\n";
    }
}
\end{verbatim}
\label{example7}
\begin{center}{Figure~\ref{example7}: example7.pl}\end{center}

\subsection{Scope, ``use strict''}
In some of the examples above the reader may have noticed the
directive 'my' used when initializing a variable.  This directive is
to declare a scope for a variable.  This scope is determined by curly
braces ( \{ \} ).  So typically the scope in our examples is for the
entire script but in some cases like the following:
\begin{verbatim} 
foreach my $v ( @list ) {} 
\end{verbatim} 
%$ 
The scope of \$v is only for the duration of the foreach loop, each
time the loop iterates a new instance of the \$v variable is created.  
If one tried to access the value of the variable after the loop a null
would be returned.  

\par Perl is a language which allows a programmer to be very lazy.  By
default all variables have the scope of the entire program.  This can
be useful for the quick hack, but disasterous for the programmer of a
large program attempting to debug a problem.  Perl does provide a way
to make the programmer more careful.  By including the module 'strict'
by simply adding \verb1use strict;1 at the beginning of your program
Perl will force the programmer to declare the scope of variable before
it can be used.  This is a very good habit to get into when writing
perl programs and will prevent a large number of subtle bugs from
being written.

\subsection{Subroutines}

Subroutines are reusuable bits of code lumped together with a defined
calling signature and should be familiar to those with programming
experience in most any language.  The word subroutine, function, and
method are used interchangeable in this document however they do mean
slightly different things to some people.  \\ 

Because perl does not have strong typing subroutines are expected
to accept 0 or more arguments in a list and return a (possibly empty)
list.  This allows for very flexible definitions and puts the burden
on the programmer to make sure their method is called correctly.
There are ways to assert the proper number of arguments are passed in
with the correct type, but these are all ultimately the responsibility
of the programmer to implement and not defined in a function signature
as in java or C.  
\par
Subroutines can be declared anywhere in a perl script, the do not need
to occur before their invokation in a script because the perl
interpreter reads in the entire perl script, compiles it, and executes
it in the interpreter rather than running each command as it is read.
Subroutines are declared by simply specifying
\verb1sub subroutinename1 and a set of braces (\{ \}) containing the
code to execute.  The implicit variable \verb1@_1 provides access to
the arguments passed in to a method.  Example 8 in Figure~\ref{example8}  
shows how to declare and use a subroutine.  The leading \& is not
necessary when calling the routine 'jump' but it helps someone reading
the code discern that 'jump' is a subroutine and not a built-in perl
function.  

\begin{verbatim}
# a random walk in 1 dimension

sub jump {
    my ($start, $size) = @_;
    my $sign = rand() > .5 ? -1 : 1;
    return $start + ( $size * $sign); 
}
my $start = 0;
print "started at $start ...";
foreach ( 1..1e5) { # iterate 10,000 times 
    $start = &jump($start,1);
}
print " ended at $start\n";
\end{verbatim}
\label{example8}
%$
\begin{center}{Figure~\ref{example8}: example8.pl}\end{center}

\section{Advanced Perl}

\subsection{References}

References in perl are a location in memory where the variable is
stored or in C terms, a pointer to the object.  References are very
useful for handling multi-dimensional arrays, storing arrays or
hashes within hashes, and for manipulating data in-place within a
subroutine or loop without having to copy the original and changed
value back and forth.

\par
The typical way to make a reference to any value in perl is to prefix
it with a $\backslash$.  Figure \ref{example9} shows a number of
different ways to interact with references.  Typically one can use a
reference to manipulate data in-place such as passing an array
reference to a function and having the data updated.

\begin{verbatim}
my @array = reverse(1..30); # 30 -> 1 
my %hash  = ( 'Jan' => 1,
	      'Feb' => 2,
	      'Mar' => 3);
my $scalar = '1900';

print "array ref is ", \@array, "\n";
print "hash ref is ", \%hash, "\n";
print "scalar ref is ", \$scalar, "\n";

my $arrayref = \@array;
my $hashref  = \%hash;
my $scalarref = \$scalar;

print "scalar value is ", $$scalarref, "\n";
print "array index 1 is ", $arrayref->[1], " index 2 is ", @$array[2], "\n";
print "hash has ", scalar keys %{$hashref}, 
    " keys and Feb value is ", $hashref->{'Feb'}, "\n";

my @one = (1,2,3);
my @two = (4,5,6);
foo(@one);
print "one is @one\n";

foo(\@two);
print "two is @two\n";
sub foo {
    my ($item) = @_;
    my $a;
    if( ref($item) =~ /array/i ) { 
        $a = $item;
    } else { $a = [ @_ ] ; }

    foreach my $v ( @$a ) {
        $v += 10;
    }
}
\end{verbatim}
\label{example9}
%$
\begin{center}{Figure~\ref{example9}: example9.pl}\end{center}

\subsection{Special Variables}

Perl has a number things that make it cryptic to novice users.  While
this may seem like a secret club, the special variables that are used
in perl are relatively straight-forward once you have a cheat sheet.

\begin{list}{$\bullet$}{}

\item \verb1$_1 - implict variable assigned during loops
\item \verb1@_1 - implict ARG variable in subroutines
\item \verb1$/1 - input record separator variable (defaults to
``\verb1\n1'')
\item \verb1$^O1 - Operating System name is stored here
\item \verb1$$1 - Process ID
\item \verb1$@1 - eval error is stored here
\item \verb1$,1 - output field separator
%$
\end{list}

\subsection{Executing external programs}

In typical Perl fashion there are a number of different ways to
execute remote programs from within a perl script.  All the methods
open a new shell on the system and execute the programs with uid of
the perl script and thus have the same path as the uid running the
script.  This can cause problems for developers who debug CGI scripts
with their own UID and then copy them to the webserver.  Sometimes a
developer will have neglected to refer to the full path for a program
or a webserver uid does not have the same access rights as the
developer.  The CGI program will then fail at these instances of
executing a remote program.
 
\par The most simple is with backticks as in the following. 
\begin{verbatim}
`/bin/cp $file1 $file2`
\end{verbatim}
However output is sent to STDOUT and not sent to the
program.  The \verb1qx//1 operator is equivalent.
\begin{verbatim} 
qx/cp $file1 $file2/
\end{verbatim}
The output from the backtick or qx operators can be
assigned to a variable.  See first example in Figure~\ref{execution}.
It is necessary to have the {\it chomp} operator in the examples
because the date function returns a line with a newline.

\par
The {\it system} command executes a command and only returns the return code
for the application rather than the output of the program.  {\it
system} takes as input an array and expects each command line option
to be a separate entry in the array.

\par A third way to execute programs and getting their output is by
handing them to the {\it open} command and executing the program
within this command.  This opens a shell on the system and runs the
command. 

\par There are other ways to execute programs using the fork command
and the IPC::Open3 to read and write I/O for a program within perl. 

\begin{verbatim}
my $date;

# get the date with backticks
chomp($date = `date`);
print "date is '$date'\n";

# get the date with qx cmd
chomp($date = qx/date/);
print "date is '$date'\n";

# get the date with the system command
$date = system("date");
print "date is '$date'\n";

open(DATE, "date |");

chomp($date = <DATE>);
print "date is '$date'\n";
\end{verbatim}
%$
\label{execution}
\begin{center}{Figure~\ref{execution}: execution.pl}\end{center}  
\subsection{Object Oriented Perl}

While Perl does not explictly force object orientation, it does have
mechanisms for introspection, inheritance, polymorhphism, and
interface defining albeit crude at times, it allows users to build
toolkits and applications with object oriented principals and expect
users of these toolkits to follow a protocol.   The following list
describes some of the mechanisms for OO programming in perl and
Example 10 in Figure \ref{example10} shows a very simple package for
ticket sales.

\begin{list}{$\bullet$}{}

\item The {\it package} directive allows the definition of a set of methods
to be associated with a particular class name.  This allows one to
build classes that interact with one another without having just a
jumble of sub routines lying around.

\item The {\it bless} directive declares an object to be of a
particular type.  Typically one {\it bless}es a hash to be particular
package and then it has access to all the methods of that package. 

\item By defining a class variable @ISA inheritance is supported.
One can have multiple inheritance, however if two parent classes have
the same function only the method from the first class listed in the
@ISA array will be derived to the child.

\item The {\it ref} method allows one to inspect what type of object
on has.  {\it ref} will return null when operating on a scalar and
will return the object reference value when inspecting a reference,
when inspecting a package blessed object it will return the name of
the package.  The associated method {\it isa} will check the @ISA array (and
all its parental dependancies) for a blessed object and see if the
queried name of a packages matches any of those used to derive the
object.
 
\end{list}

Much more information is available about OO perl in Damian Conway's
book {\it Object Oriented Perl}

\begin{verbatim}
# code to use the package

my $seller = new TicketSeller(10);
my $total = 0;
while ( $seller->buy_ticket ) {
    print "Bought a Ticket\n";
    $total++;
}
print "$total tickets sold\n";

# package defined
package TicketSeller;

use strict;

sub new { 
    my ($class, @args) = @_;
    my $self = {
       'numtickets' => shift @args	
    };
    bless $self, $class;
    return $self;
}


sub buy_ticket {
    my ($self) = @_;
    if ($self->{'numtickets'} > 0) {
         $self->{'numtickets'}--;
         return 1;
    } 
    return 0;
}

1;

\end{verbatim}
\label{example10}
%$
\begin{center}{Figure~\ref{example10}: example10.pl}\end{center}

\newpage
\section{Resources}

\subsection{Books}

\par
There are a number of good perl books out there.  I will list a few of
the ones that are good starters.  The O'Reilly series is usually good,
but may not provide enough detail for novice programmers.  Damian
Conway's book is not for the faint of heart
 
\begin{list}{$ \bullet $}{}
\item {\it Learning Perl}, Randal Schwartz and Tom Phoenix.  O'Reilly
\& Associates. 3rd edition July 2001.

\item {\it Programming Perl}, Larry Wall, Tom Christiansen, \& Jon
Orwat. O'Reilly \& Associates.  3rd edition May 2000.

\item {\it Advanced Perl Programming}, Sriram Srinvasan. O'Reilly \&
Associates. 1997.

\item {\it Object Oriented Perl}, Damian Conway.  Manning
Publications. 1999.  

\end{list}


\subsection{WWW}
\par

A number of great online tools exist for learning perl and programming
for bioinformatics.

\begin{list}{$ \bullet $}{}
\item \url{http://stein.cshl.org/genome_informatics/} - Lincoln Stein's CSHL Genome Informatics course has many of the lectures and references
linked online.

\item \url{http://www.perl.com} - O'Reilly's everything perl site.
Some reasonable tutorial links are there

\item \url{http://bio.perl.org} - Bioperl site has links about
bioinformatics and perl.

\item \url{http://www.comp.leeds.ac.uk/Perl/start.html} Nik Silver's
page - very basic, but much like this document.

\item \url{http://www.netcat.co.uk/rob/perl/win32perltut.html} Robert
Pepper's Perl Tutorial including some Win32 specifics.
 
\end{list}
\end{document}
